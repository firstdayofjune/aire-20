\section{Introduction} % (fold)
\label{sec:introduction}
\colorbox{yellow!30}{ToDo: Find \& add references}\\
The result of a good software is depends on a lot more than the pure engineering.
Software is designed to perform and automate a task, which otherwise would have to be executed manually.
In order to fulfill the given task properly, it is necessary to completely understand the task's underlying problem. One strategic approach to develop this understanding is to conduct a thorough analysis of the requirements (of both the business and the software) and to document the results according to given standards. This process is called requirements engineering (RE). The work products of the RE process can be manifold. In the scope of this research study we will therefore focus on the actual requirements.\\

One part of the RE process is to collect, order and prioritize the requirements. Depending on the number of requirements, it may help or even be necessary to use some data analysis techniques to automatically derive some useful insights, which can then be used in the further decision-making process.
In this paper, we will have a look at the requirements which were previously collected by ... in the \crowdre{} project. From the perspective of a fictitious product owner we want to answer the following question:\\
\begin{quote}
\textit{Given a set of requirements, what kind of features are our potential customers interested in the most?}
\end{quote}

We consider our product owner to be working in a company which builds smart home appliances and deem the \crowdre{} requirements to be the result of a survey this company has performed. The collected requirements therefore are the foundation of our analysis.\\
Considering the sheer size of the requirements, we don't want to go through each of the requirements manually, but want to setup and use an unsupervised neural network to perform the analysis. We will use the Python programming language to do so and want to derive a set of categories where the collected requirements fall into.\\
Finally, we want to answer our initial question based on the categories we found and the number of requirements assigned to each of them.
% section introduction (end)