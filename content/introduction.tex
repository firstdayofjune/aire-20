\section{Introduction} % (fold)
\label{sec:introduction}
\blockcomment{
	The result of a good software is depends on a lot more than the pure engineering.
	Software is designed to perform and automate a task, which otherwise would have to be executed manually.
	In order to fulfill the given task properly, it is necessary to completely understand the task's underlying problem. One strategic approach to develop this understanding is to conduct a thorough analysis of the requirements (of both the business and the software) and to document the results according to given standards. This process is called requirements engineering (RE). The work products of the RE process can be manifold. In the scope of this research study we will therefore focus on the actual requirements.\\

	One part of the RE process is to collect, order and prioritize the requirements. Depending on the number of requirements, it may help or even be necessary to use some data analysis techniques to automatically derive some useful insights, which can then be used in the further decision-making process.
}
%% Section Outline:
The success of software projects depends on more than just software engineering\,\cite{mohagheghi_what_2017}. Requirements engineering (RE) e.g. is considered the key success factor in software projects\,\cite{mavin_towards_2019}. Also, the success of software products \textit{``often  depends  on  user  feedback''}\,\cite{maalej_toward_2016}. In traditional RE, techniques like surveys, workshops, observations and interviews are used to gather user feedback and to elicit software requirements\,\cite{pohl_requirements_2015}. Usually, these techniques are limited and can only be applied to end-users within organizational reach\,\cite{oriol_fame_2018}. With the emergence of new data sources this changes: Researchers have shown different approaches on how to extract requirements from e.g. tweets or app store reviews\,\cite{oriol_fame_2018,stanik_classifying_2019} and in 2016, Murukannaiah et al. used crowd sourcing to elicit about 3000 requirements for smart home applications\,\cite{murukannaiah_acquiring_2016}. Such forms of user feedback can be used to \textit{``identify, prioritize and manage the requirements''}\,\cite{maalej_toward_2016} for software products and to increase user satisfaction\,\cite{palomba_user_2015}. However, \textit{``automated techniques are necessary to derive useful insights from large amounts of raw data the crowd can produce''}\,\cite{murukannaiah_toward_2017}. This becomes even more apparent, as the decision-making process in requirements engineering shifts towards a more data-driven approach\,\cite{maalej_data-driven_2019}. The automatic analysis of crowd based requirements comes with some challenges, though, as Murukannaiah et al. declared in \,\cite{murukannaiah_toward_2017}. With our paper, we work on the second of these challenges, on how to summarize crowd-acquired requirements (DC2). Our contribution is a method to cluster the aforementioned smart home requirements through the combined use of topic modeling techniques and similarity metrics based on word embeddings.
% section introduction (end)