\section{The CrowdRE Dataset}
- An approach towards scaling the RE process through the engagement of the general public. 

- Necessary to use automated techniques to gain useful insights 

- The dataset contains RE for smart home appliances 

- Reqs were submit by 609 Amazon Mechanical Turk users (https://www.mturk.com/) 

- Furthermore, the personal characteristics of the crowd workers who supplied the RE are recorded “including their demographics, personality, traits, and creative potential”, as gathered in a presurvey (questions, Mini-IPIP scale, CPS)  

- It was an attempt to “facilitate large scale user participation in RE” \cite{murukannaiah_toward_2017}

\subsection{Challenges / Motivation / Benefits}
Crowd RE made it possible to gather a large amount of data 

Raw data is of little use, but to derive information from the data manually may be difficult and is error prone, e.g. when looking at the sheer amount of information gatherted 

Also, human effort is a cost factor and the time is better spent on tasks which can not be automated, yet 

We, as the authors, can be very happy to base our research on the Crowd RE dataset, as it is quite cumbersome to curate data which can be used to train and test automated techniques 

Authors of the Crowd RE reqs already tagged their reqs into the domains Energy, Entertainment, Helath, Safety, Other -> Could be used for verification