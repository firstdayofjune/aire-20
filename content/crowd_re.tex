\section{The \crowdre{} Dataset}
In an attempt to “facilitate large scale user participation in RE” \cite{murukannaiah_toward_2017} 609 Amazon Mechanical Turk users\footnote{\url{https://www.mturk.com/}} were asked to submit requirements for smart home appliances in the \crowdre{} project. The requirements were collected through a form and the submissions were gathered in a database. The resulting data consisted of 2966 requirements, related to the domains \emph{Energy, Entertainment, Health, Safety} and \emph{Other}.\\

The requirements were collected in two phases: In the first phase the crowd workers were asked for their requirements of a smart home. The phase comprised three stages in which the workers were given a number of requirements and they were asked to add 10 requirements which are distinct to what they have seen. To ensure the requirements sentences follow the user story format\footnote{As a [role] I want [feature] so that [benefit].}, the form was separated into one field each for the role, the feature and the expected benefit. Furthermore, one of the aforementioned domains had to be selected as the \emph{application domain} of the requirement. Finally, an arbitrary number of tags could be added to the requirement. Unlike with the domains, the tags were not given and had to be defined by the users themselves. The resulting requirement would then look as follows\footnote{The keywords marked in bold text in the requirements sentence represent the placeholders which were already provided by the form to preserve the user story format.}:

\begin{tabbing}
Requirement: \= \textit{``\textbf{As a} pet owner, \textbf{I want} my smart home to let me know when}\\
\>\textit{the dog uses the doggy door, \textbf{so that} I can keep track of the}\\ \>\textit{pets whereabouts.''}\\
Domain:\> \textit{Safety}\\
Tags:\> \textit{Pets, Cats, Dogs}
\end{tabbing}

In the second phase, the crowd workers were presented with the requirements produced in the first phase and they were asked to rate the requirements with regard to their clarity, usefulness and novelty. Note that for our analysis though, we only rely on the results of phase one. The second phase is mentioned for the sake of completeness.