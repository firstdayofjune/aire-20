\section{Conclusion} % (fold)
\label{sec:conclusion}

To sum up our research we can say that the used approaches work much better on large data sets and our dataset is compared to common used ones small. Also the soft labeling of the data that was done by the users themselves doesn't have a good quality. To increase the quality of the labeling manual checking with a lot of effort needs to be performed. The bad quality of the labeling might also result from a missing common understanding of the domains within the crowd workers.

In the end still a lot of manual work needs to be done to derive valuable results from the dataset regarding what features or feature categories are wanted the most in smart home application.

Finally we can summarize that the idea of CrowdRE works to some extend. One can gather a lot of requirements that are assigned to the smart home topic, but it is very difficult to analyze them automatically and requires a lot of effort to prepare them for further work.

% section conclusion (end)