\subsubsection{Word Mover's Distance} % (fold)
\label{sub:word_movers_distance}
While word2vec is very sophisticated when it comes to generating quaility word embeddings, the CBOW method still has its weaknesses. Consider the two documents: \emph{"My smart home should turn on my favorite music when I come to my home."} and \emph{"My smart home shall play my most favored songs when I arrive at my place."} Even though the information is the same, the word vectors of these sentences will be different. Even though a word-wise similarity will be given (e.g. of the pairs $< music, songs>$, $<come, arrive>$) the closeness of the sentencens can not be represented by the CBOW model. To overcome this shortage, Kusner et al. introduced the Word Mover's Distance (WMD) in 2015 \cite{kusner_word_2015}. The WMD is a distance function which can be used to calculate the distance between these kind of text documents. Based on previously created word embeddings (as for example using the word2vec), the \textit{"distance be- tween[sic!] two text documents A and B is the minimum cumu- lative[sic!] distance that words from document A need to travel to match exactly the point cloud of document B"}\cite[p2]{kusner_word_2015}. Using this method, the WMD reaches a high retrieval accuracy, while being completely free of hyper-parameters and therefore straight-forward to use.