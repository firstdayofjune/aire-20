\section{Proposed Approach} % (fold)
\label{sec:own_approach}

The \crowdre{} dataset is available in form of a MySQL database dump, but the tables can also be downloaded separated into several \textit{.csv} files \cite{crowdre_dataset}. For our research, we were only interested in the pure requirement sentences (without any ratings, or user characterization added to the data). We therefore reconstructed the sentences from the \textit{requirements.csv} file, which is included in the downloaded data.

To have some measure to evaluate the proposed approach we need a labeling at the dataset that we can use to rate how good the topic modeling worked. At first we checked if we can use the user defined tags as soft labeling for the requirements. Unfortunately most of them are only matched once (Total tags: 2116, tags that only occur once: 1562). Additionally we evaluated the most common used tags and the coverage\footnote{specific number of tags covering a significant amount of requirements} of the requirements.

\begin{figure}[ht]
  \centering
    \includegraphics[width=0.8\textwidth]{figures/tag_analysis.pdf}
    \caption{Tag occurrence and coverage of the requirements}
    \label{fig:tag_analysis}
\end{figure}
\FloatBarrier

In \autoref{fig:tag_analysis} the coverage of requirements by the given tags is shown. The fact that about $\frac{1562}{2116}\approx73.8\,\%$ of the tags only occur once leads to a small coverage of requirements by the given tags. The variety of tags that may be assigned to the same topic is very high and the low coverage of requirements with the top 9 tags makes the tags not suitable for the soft labeling. 

Another approach to get a labeling for the evaluation was to check the domains that were assigned to the requirements. The domains are separated into five groups: Health, Energy, Entertainment, Safety and Other. For the \grqq{}Other\grqq{} there are again user defined specific domains, but we focus on the five top level domains for our labeling.

\subsection{NLP Preprocessing Pipeline} % (fold)
\label{sub:own_pipeline}

\begin{figure}[ht]
  \centering
    \includegraphics[width=\textwidth]{figures/NLP Pipeline.pdf}
    \caption{Processing an exemplary requirement sentence through our NLP Preprocessing Pipeline.}
    \label{fig:nlp_pipeline}
\end{figure}

As initially described in \autoref{sec:nlp} we preprocessed our requirement documents using an NLP pipeline as shown in \autoref{fig:nlp_pipeline}. Implementing our solution in Python and following the common practice as suggested in \cite{ferrari_natural_2018}, we made use of the NLTK library \cite{nltk_library} to perform the NLP techniques needed for our analysis. As some of the requirements sentences contains special characters, some initial data cleansing was necessary, to remove these special characters (e.g. spaces, dots, apostrophes, slashes) as they would have otherwise been ranked in the later used bag of words. We used regular expressions as provided by the Python standard library in order to do so. For the tokenization, the stop-word-removal and the stemming we used the functions provided by the NLTK API.
% subsection preprocessing (end)
\subsection{LDA Approach} % (fold)
\label{sub:own_lda}

After we developed our pre-processing pipeline for the dataset and some basic analysis on the data we have we decided to use the LDA for a first topic modelling. The LDA approach serves as reference for the result we wanted to obtain by the neural network to have a result for evaluating.


For our LDA approach we used our pre-processed requirements. To apply the LDA we transformed the data in the following way. We created a matrix where each row represents one of the 2966 requirements. the columns are the single words of the requirements. But as the LDA needs a numerical representation of the words we first applied a bag-of-words to the single requirements. As the results were not sufficient we decided to calculate the TF-IDF to get weights for the single words. After these steps we had a prepared matrix that holds the data that now can be used for the LDA.

\begin{figure}[bht]
    \subfigure[LDA Plot with colors to topics\label{fig:lda-tf-idf-topics}]{\includegraphics[width=0.49\textwidth]{screenshots/lda-tf-idf.pdf}}
    \subfigure[LDA Plot with colors for domains\label{fig:lda-tf-idf-domains}]{\includegraphics[width=0.49\textwidth]{screenshots/lda-domains.pdf}}
    \caption{LDA Result with TF-IDF (plotted with t-SNE)} \label{fig:lda-tf-idf}
\end{figure}
\FloatBarrier

In \autoref{fig:lda-tf-idf} we can see the two dimensional representation of the results of the LDA. The reduction of the dimensions is performed by t-SNE which tries to preserve the most differing dimensions. In \autoref{fig:lda-tf-idf-topics} the colors are mapped to the found topics which leads to separable clusters. But if we look for the expected mapping to the domains in \autoref{fig:lda-tf-idf-domains} we can see that the found cluster doesn't represent the expected clusters that where defined by the domains.
\subsection{Word Embeddings}
In order for our approach to also respect the semantic regularities of the requirements sentences, it is not enough to rely on the clusters generated by the LDA, as mentioned in \autoref{sub:back_word_embeddings}. We therefore create word embeddings, using the techniques we described earlier.

\subsubsection{word2vec} % (fold)
\label{sub:own_word2vec}
Besides offering implementations of Latent Semantic Analysis and LDA, the also implements the word2vec tool created by Mikolov et al. and was recommended for the generation of word embeddings in a review on NLP toolkits performed in 2018\,\cite{solangi_review_2018}. With word2vec it is possible to train a natural language model (either based on the CBOW or the skip-gram architecture) on a text corpus, to learn the word embeddings for the words contained in the corpus. As shown in \autoref{fig:w2v-pipeline}, we therefore use our corpus of requirements sentences derived from the \crowdre{} dataset to train a language model on both architectures, with and without processing the requirements through our NLP pipeline before.
\begin{figure}[ht]
  \begin{center}
    \includegraphics[width=0.9\textwidth]{figures/word2vec_pipeline.pdf}
    \caption{Training a language model on our requirements corpus using word2vec.}
    \label{fig:w2v-pipeline}
  \end{center}
\end{figure}
\FloatBarrier

Being unsatisfied with the outcome, in a second attempt we created our embeddings using a pre-trained word2vec model, which contained the word vectors of a model trained on about 100 billion words of the Google News dataset\footnote{\url{https://code.google.com/archive/p/word2vec/}, last visited 2020-01-19}. Even though the outcome was different, the underlying workflow for both approaches was very similiar once the trained model was available:
\begin{itemize}
	\item For every tokenized requirement sentence, create a sentence matrix by replacing every word by its vector representation:\\
	$reqtokens = \{"As", "smart", "home", "owner", \dots\}$\\
	$embeddings = \{ \vec{as}, \vec{smart}, \vec{home}, \vec{owner}, \dots \}$
	\item On the resulting matrices, reduce the different x-dimensions to the dimension of the shortest sentence using Principal Component Analysis
	\item Use K-Means to generate a number of clusters on these now equally shaped matrices
	\item Visualize the results by transforming the data into 2d space using t-SNE\cite{maaten_visualizing_2008}
\end{itemize}


\begin{figure}[ht]
  \begin{center}
    \includegraphics[width=\textwidth]{screenshots/google_word_2_vec_tsne_opti4.png}
    \caption{word2vec result of the clustering with a pretrained model (plotted with t-SNE)}
    \label{fig:w2v-pretrained-4}
  \end{center}
\end{figure}
\FloatBarrier
\subsubsection{Word Mover's Distance} % (fold)
\label{sub:own_wmd}
Finally, we used the Word Mover's Distance which was also implemented in the gensim library. To do so, it was also necessary to create word embeddings for our requirement sentences first. Again, we could use the word vectors of the aforementioned word2vec models. Instead of basing our clusters on the distance between word vectors though, we could now calculate a distance matrix which holds the calculated Word Mover's Distance from every sentence to every other sentence, as represented in \autoref{tbl-wmd-matrix}. Note how the Word Mover's Distance is symmetric. So the distance to travel from $s_1$ to $s_2$ is the same as if your travelling vice versa.

\begin{table}[ht]
\centering
\begin{tabular}{lllll}
             & $s_1$                      & $s_2$                      & $s_3$                      & ...                      \\ \cline{2-5} 
\multicolumn{1}{l|}{$s_1$} & \multicolumn{1}{l|}{0}    & \multicolumn{1}{l|}{0.83} & \multicolumn{1}{l|}{3.23} & \multicolumn{1}{l|}{...} \\ \cline{2-5} 
\multicolumn{1}{l|}{$s_2$} & \multicolumn{1}{l|}{0.83} & \multicolumn{1}{l|}{0}    & \multicolumn{1}{l|}{2.77} & \multicolumn{1}{l|}{...} \\ \cline{2-5} 
\multicolumn{1}{l|}{$s_3$} & \multicolumn{1}{l|}{3.23} & \multicolumn{1}{l|}{2.77} & \multicolumn{1}{l|}{0}    & \multicolumn{1}{l|}{...} \\ \cline{2-5} 
\multicolumn{1}{l|}{...}  & \multicolumn{1}{l|}{...}  & \multicolumn{1}{l|}{....} & \multicolumn{1}{l|}{...}  & \multicolumn{1}{l|}{0}   \\ \cline{2-5} 
\end{tabular}
\caption{Sentence Matrix containing the Word Mover's Distance from one sentence to another}\label{tbl-wmd-matrix}
\end{table}

 Since by its nature, the resulting matrix already was a 2-dimensional array with equal dimensions, it was not necessary anymore to perform any further reduction. We could use this matrix instead to directly create some clusters using K-Means.

 \begin{figure}[ht]
  \begin{center}
    \includegraphics[width=\textwidth]{screenshots/our_word_movers_distance_tsne.png}
    \caption{Distance Matrix of the Word Mover's Distance with a self-trained model (plotted with t-SNE)}
    \label{fig:wmd-selftrained-1}
  \end{center}
\end{figure}
\FloatBarrier
